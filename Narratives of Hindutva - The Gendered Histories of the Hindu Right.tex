\documentclass[a4paper, 12pt]{article}
\usepackage[protrusion=true,expansion=true]{microtype}
\usepackage{indentfirst,csquotes,graphicx,url,endnotes}
\usepackage[T1]{fontenc}
\graphicspath{{C:/Users/vmk3/OneDrive/Pictures/}}
\linespread{1.05}
\makeatletter
\renewcommand*\makeenmark{\hbox{\textsuperscript{\@fnsymbol{\theenmark}}}}
\renewcommand{\maketitle}{
\begin{flushright}
{\LARGE\@title}\\
\vspace{50pt}
{\large\@author}
\\\
\vspace{40pt}
\end{flushright}
}
\title{\textbf{Narratives of Hindutva}\\The Gendered Histories of the Hindu Right}
\author{\textsc{Vishnu Kumar}\\{\textit{Rice University}}}
\begin{document}
%\maketitle
%\section*{2002 Gujarat and the Histories of the Hindu Right}
%In April of 2002, not two months after the beginnings of the anti-Muslim pogroms in Gujarat, then-prime minister Atal Bihari Vajpayee gave a speech to a gathering of his Bharatiya Janata Party (BJP) members that essentially justified the carnage by insinuating that Muslims in Godhra started the violence when they immolated a train (the Sabarmati Express) that had been carrying Hindu pilgrims: \begin{displayquote} What happened in Gujarat? If the conspiracy to burn alive the innocent, helpless, and blameless travelers on the Sabarmati Express had not been hatched, the Gujarat tragedy could have been averted. But this did not happen. People were burnt alive. Who were those people?... The later incidents are condemnable, but who started the fire?\footnote{Atal Bihari Vajpayee, `"Who Started the Fire?": Rough Literal Translation of the Full Text of the Controversial Speech given by the Prime Minister at BJP’s National Executive Meeting in Goa on April 12, 2002,’ \textit{Outlook}, April 2002.}\end{displayquote} In an interview conducted barely 10 days after the violence had largely abated, then-Chief Minister of Gujarat and current Indian Prime Minister Narendra Modi hastened to identify the train fire specifically as a terrorist act while keeping the pogroms that happened in its wake under the rubric of communal violence, for both rhetorical and legal reasons: \begin{displayquote} \textbf{Q:}The state government is giving compensation of Rs 2 lakh to next of kin of the victims of the Godhra carnage and Rs 1 lakh for those killed in the communal flare-up afterwards?\par \textbf{A:} The Godhra incident was not communal violence but terrorism. It was a Congress MLA who in 1992 moved a resolution in the state assembly for giving Rs 1 lakh compensation to the victims of communal violence. The compensation is being given according to the law.\footnote{"`My Government is Being Defamed,' says Modi," \textit{The Tribune}, 10 March 2002}\end{displayquote} Subsequent forensic reports indicate that the fire not only began from inside the train but was also most likely accidental.\footnote{‘Gujarat Riot Death Toll Revealed’, BBC News, 2005 <\url{http://news.bbc.co.uk/2/hi/south_asia/4536199.stm.}>} The fact that at least 790 people were killed in the post-train fire violence in comparison to the 59 who died in the actual fire shows just how frenzied the revenge violence was.\footnote{‘Gujarat Riot Death Toll Revealed’} 
%\par
%The important thing about statements like Vajpayee's and Modi's is that they are not defensive explanations made after the fact trying to make sense of what happened. Instead, they actively historicize and concretize the events into a truth of Muslim domination over the Hindu majority. The forensic report casting the Sabarmati fire as accidental came out before Vajpayee's speech, so this active historicization is not one that concerns itself with the ground-level facts. Hindu nationalists are able to play fast and loose with the facts of this specific incident because \textit{it does not stand alone} in their narrative of Hindu subjugation, as is made clear further on in Vajpayee's speech. He places the Sabarmati train fire in a long line of perceived Muslim injustices toward Hindus both past: \begin{displayquote} …no [Hindu] king, at the time of the attack, destroyed any temple or broke any idol. This is our tradition of treating all religions equally. But still, charges are being levelled that secularism is under threat. Who are these people levelling these allegations? What does ‘secularism’ mean for them? When the Muslims had not arrived in India…even then India was secular. It is not that it’s become secular after their arrival…\footnote{Vajpayee}\end{displayquote} and present: \begin{displayquote} We have been fighting against terrorism for 20 years. The terrorists tried appropriating Jammu and Kashmir by violent means but we fought back.\footnote{Vajpayee}\end{displayquote}
%\par 
%This skillful weaving of isolated historical incidents into a crystallized historical narrative of Muslim oppression of the Hindu majority has been the centerpiece of Hindu nationalist ideology from its very beginning. Its fulcrum is the specter of a male Muslim “other” inflicting violence and suffering upon dignified yet helpless Hindus, especially Hindu women.\footnote{Kalyani Devaki Menon, \textit{Everyday Nationalism: Women of the Hindu Right in India} (University of Pennsylvania Press, 2010). 26-27} Crucially, this gendered and highly sexualized historiography is not just a theoretical crutch but a practical rallying cry on the ground for mobs purporting to protect Hindu honor. To see how this theoretical historiography manifests as actionable, ground-level rhetoric, it is worth quoting in full a pamphlet distributed at the time of the pogroms by the Vishva Hindu Parishad, a right-wing organization belonging to the umbrella of Hindu nationalist organizations known as the Sangh Parivar. The pamphlet is a rather amateurish poem in Gujarati entitled “Jihad:”\begin{displayquote}The people of Baroda and Ahemedabad have gone berserk\\Narendra Modi you have fucked the mother of \textit{miyas}\\The volcano which was inactive for years has erupted\\It has burnt the ass of \textit{miyas} and made them dance nude\\We have untied the penises which were tied till now\\Without castor oil in the ass we have made them cry\\Those who call religious war violence are all fuckers\\We have widened the \textit{bibis} tight vaginas\\ \dots\\ Wake up Hindus, there are still \textit{miyas} left alive around you\\Learn from Panwad village where their mother was fucked\\ She was fucked standing while she kept shouting\\ She enjoyed the uncircumcised penis\\ With a Hindu government, Hindus have the power to annihilate \textit{miyas}\\Kick them in the ass to drive them... out of the country.\end{displayquote} The relentlessly sexual imagery in the poem harkens back to an ancient obsession with the apparent virility of the meat-eating, animal sacrificing Muslim male as compared to the Hindu male.\footnote{Christophe Jaffrelot, ‘The 2002 Pogrom in Gujarat: The Post-9/11 Face of Hindu Nationalist Anti-Muslim Violence’, in \textit{Religion and Violence in South Asia: Theory and Practice}, ed. by John Hinnells and Richard King (Routledge, 2007), pp. 173-92.} Unfortunately, the poem’s imagery is an accurate reflection of what happened to women in the pogroms. Several survivors have recounted the same basic patterns of violence: rape, gang-rape, sodomy, burning alive, and infanticide.\footnote{Tanika Sarkar, ‘Semiotics of Terror: Muslim Children and Women in Hindu Rashtra’, \textit{Economic and Political Weekly}, 37.28 (2002), 2872-76.} Though one could go on talking about the many ways in which women and children were raped and killed in the pogroms, three important points must be made: the torture of women’s bodies was endlessly inventive, reproductive organs were attacked with especially intense savagery, and children and the unborn were not spared. 
%\par
%Here it feels necessary to pause and ask what purpose the excessive cruelty displayed in those pogroms serves. While the violence in general fits into the rhetoric of action and revenge that is the hallmark of Hindu nationalism, it would seem as though excessive, gratuitous violence would have no further benefit. To fully answer this question, we must look at the histories that Hindu nationalist groups have spread throughout India. There we can find a myriad of tales involving Hindu women raped by sexually frenzied Muslim men. Undergirding these tales is an insecurity, a penis envy, that can only be fixed through violence. In this way, violence for Hindu nationalist becomes proof of their masculinity. This can be seen in the chants of the student wing of the BJP known as the Akhil Bharatiya Vidyarthi Parishad (ABVP) immediately following the Godhra train fire: "Those Hindus whose blood does not boil are not Hindus but eunuchs."\footnote{Sarkar} Thus, for the Hindu nationalist, taking back India means making Hindu male sexuality as predatory as their self-constructed version of Muslim male sexuality. 
\section*{"Our" Women}
The preceding discussion of the spectacular levels of sadism inflicted upon Muslim women during the 2002 riots leaves open the question of the role of Hindu women in Hindu right-wing nationalism. While the calls to uninhibited violence, rape, and bloodshed may indeed be attractive to a certain frustrated young Hindu male, it is difficult to see how the same discourse would encourage Hindu women to come out and support the Sangh (the general umbrella term for right wing Hindu organizations). However, it is evident that they do, for no social movement can be successful if it relies solely on one small facet of a nation’s citizenry. Indeed, the Sangh's widespread success in India can only be explained by an ability to appeal to many disparate Hindu demographics. 
\par
Earlier in this essay, we looked at the Hindu nationalist historiography disseminated by BJP and/or Sangh leaders. It is important to note that those party leaders are almost invariably male, making their focus on male sexual humiliation understandable. However, a similar focus would most likely not appeal to Hindu women on the same visceral level. Thus, we can find that the histories disseminated amongst the women of the Hindu right differ in focus than the histories disseminated amongst the men of the Hindu right. These histories then, are key to understanding the expansion of right-wing Hindu ideology. Importantly, these histories are not necessarily \textit{opposed} to the more well-known histories disseminated amongst the men making up the majority of Hindu nationalism's public face since there must be some common core of ideological stability. What I hope to show in this section of the essay is the means by which these histories both expand the right-wing Hindu base and reiterate the “classical” right-wing Hindu historiography (that is, of Muslim invasion and domination over the native Hindus) outlined above. 
\par 
The main difference between the normative histories of the Hindu right and what I will call the gendered histories of the women of the Hindu right is the status of women as victims of feral Muslim men in the former and their status as actionable agents of history in the latter. In fact, they are not just agents of history but agents in the destiny of the Hindu nation. In the stories that I will focus on from the female counterpart of the Rashtriya Swayamsevak Sangh (RSS), the Rashtra Sevika Samiti (henceforth the RSeS) this is done by making motherhood the locus of the female capacity to effect historical change. This is best seen in the stories told by the RSeS about Jijabai, the mother of the legendary 17th century king Shivaji (a hero to the RSS). What follows is my own brief paraphrasing of several RSeS pamphlets - orignially in Hindi - telling of a pivotal event in the young Shivaji's life. \begin{displayquote} One day when Shivaji was a boy, he was playing \textit{chauser} with his mom, Jijabai. He asked: "Mother, what must I do to win?" From where they were sitting, they could see a hill in the distance, on top of which flew a green flag. Pointing to it, Jijabai answered "That flag must turn saffron."\footnote{‘Matritva Ka Aadarsh Jijamata (Mother Jijabai: A Model of Motherhood)’, \textit{Rashtra Sevika Samiti}, (Uttaranchal, January 1999), pp.7-9.}\end{displayquote} The color coding seen here is made explicit later on in the print source of this story, where the antagonists are explicitly identified as Muslim. One of the most important things about this story is its malleability to the audience. There are two parts of the narrative that the teller of this story can focus on, depending on their audience: the mother's role in shaping the next generation's worldview or the metaphorical distance of the green flag (i.e., Muslims) from the obviously Hindu mother and child. If the speaker is uncertain of the audience’s sympathy towards Muslims or of their historical awareness, they can choose to emphasize the motherhood aspect of the story to show how women can participate in nation building in their daily lives simply through how they raise their children. However, if the speaker is confident that the audience's antipathies and thirst for action, they can choose to emphasize the foreignness of the Muslims and the unity of the Hindus under the saffron flag. 
\par 
Before going further, it is worth noting how this very strategy inherently reiterates the same patriarchal power structures present throughout the nationalist movement. Placing motherhood as the site for female agency and worth calls back to a long tradition of how women are conceived in conceptions of the nation. As Anne McClintock puts it, "all nationalisms are gendered, all are invented, and all are dangerous."\footnote{Anne McClintock, ‘Family Feuds: Gender, Nationalism, and the Family,' \textit{Feminist Review}, 44 (1993), 61-80.} The nature of nationalist rhetoric inherently rests on the erection of borders between nation-states and therefore, ideally for nationalists, between ethnicities. In this conception of nationalism, women play a part not so much as the citizenry of whatever nation-state is being conceptualized but as a border guard of sorts. Follwing this line of thought, Nira Yuval-Davis and Floya Anthias identify five roles that women play in nationalist discourse: \begin{displayquote}\begin{itemize}\item Biological reproducers of the members of national communities\item Reproducers of the boundaries of national communities through restrictions on marital and sexual relations\item Active producers and transmitters of the national culture\item Symbolic signifiers of national difference\item Active participants in national struggles\footnote{\textit{Woman-Nation-State}, ed. by Nira Yuval-Davis and Floya Anthias (Palgrave Macmillan, 1989), 7.}\end{itemize}\end{displayquote} It is striking that at least three of these five points can be tied back to motherhood. It is obviously true for the first point, the second point can basically be rephrased to say that the boundaries are set through who women may and may not have children with, and the third point basically means that mother should inculcate her (preferably male) children with nationalist values. These attitudes are not recent in nationalist discourse. For example, in the words of John Ruskin:\begin{displayquote}The man’s power is is active, progressive, defensive. He is eminently the doer, the creator, the discoverer, the defender. His intellect is for speculation and invention; his energy for adventure, for war, and for conquest… But the women’s power is for rule, not for battle, and intellect is not for invention or creation, but for sweet ordering, arrangement, and decision…she is protected from all danger and temptation…he [man] guards the woman from all this, within his house, as ruled by her.\footnote{John Ruskin, ‘Of Queen’s Gardens’, in \textit{Sesame and Lilies}, 1865.}\end{displayquote} In other words, the woman plays a passive role in the construction of a nation, her role being an anthropomorphization of national boundaries. 
\par
The fourth point, if not directly tied to reproduction, still places the characteristics of the nation passively on the female body. By placing motherhood as the ideal form of female agency, the RSeS takes it upon themselves to place the “maintenance of authenticity, originality, purity, and virtue” on the female body even as men take “the powers of the public and visibility of the political."\footnote{Sumathi Ramaswamy, \textit{The Goddess and the Nation: Mapping Mother India} (Duke University Press, 2010). 75.} In this particular case, it seems as though the Sangh's discourse is trying to bring even the fifth point into the fold. That is, the Sangh seems to be trying to redefine motherhood not just as a passive reproduction of national boundaries but as an active part of nation-building. Though this may seem subversive, the heterodoxy of these RSeS stories should not be overstated, for they still need to be woven into the larger more singular vision of Hindu nationalist ideology. For example, here is part of the text of a speech given by Usha Chati, an RSeS leader, in 1999:\begin{displayquote} Ever since the Muslim invaders came to India, there have been atrocities committed against women and, consequently, society was no longer a safe place for them. In ancient India, Hindu society gave women a lot of freedom and there was no parda and women could wander alone with no fear. All this has changed with Muslim invasions and now women have to stay at home, maintain parda. Because in those days if the Muslims saw women and like the way they looked, they would just pick them up and force them to become part of the harem. Ever since those times, women have been unsafe and even today a woman out of doors is not safe. This is why the Samiti teaches women to defend themselves and also gives them mental strength to overcome their fear.\footnote{Usha Chati, Speech at Dharam Bhavan, Nov 25, 1999}\end{displayquote} The deference to normative Hindu right-wing historiography in this speech cannot be missed. Though this speech was given 19 years ago, the rhetoric about women’s safety still rings true today, especially in Delhi (where this speech was given), showing the adaptability of this kind of rhetoric demonizing the “other.” One point of difference between this nationalist history and the nationalist history promulgated by the RSS or VHP is its focus on the everyday safety of women. Instead of placing their story of Muslim predation in a grand, sweeping narrative of historical Hindu subjugation, the RSeS grounds it in a way that would appeal to the average woman’s everyday fear of danger on the street. In this way, the message is softened, with the call to action being not hardline violence but self-defense. This makes recruiting women into the right-wing fold much easier since they have now been convinced that doing so would be beneficial for their, and by extension their families’, well-being. 
\section*{Conclusion and Further Research}
The role of women in the Hindu right-wing is primarily that of a victim in nationalist history, with their victimhood used to spur on the energies of mostly young men towards exacting often violent “revenge” on Muslims. However, this conception of nationalist duty would not seem to appeal to the average Hindu woman, which would severely hinder the movement’s expansionary potential. Thus, the women of the movement, such as the members of the RSeS, have had to find new ways to retrofit the normative Hindu nationalist histories to appeal to the average women. They have done so by placing the fear of the “other,” in this case Muslim men, into the context of the everyday. In doing so, not only is the usual message of hardcore violence and revenge softened but the fear is made latent, available to be inflamed at a moment’s notice, as in Gujarat in 2002. The fulcrum around which all of this turns is a deliberate cherry-picking and distortion of history. In this respect, the Hindu right is hardly unique among global nationalist movements: a quick glance at any of Donald Trump’s speeches will confirm that. However, understanding how the Hindu right appeals to women, and the specific historical distortions it uses, is crucial to understanding the movement itself and how to defeat it.
\par
	Stepping back, there is another potential line of work that this analysis opens up. I examined how the histories of the Hindu right wing are not as rigid as they might seem upon first glance, adaptable as they are to different demographics. However, we can now see in India how Hindu nationalist histories adapt themselves to more than just gender. BJP-RSS violence is on the rise in Kerala, a small state in Southwestern India that has long been a bastion of communism, literature, and secularist thought. In short, it is a state seemingly inhospitable to communalist violence. Ground realities point the other way, with O. Rajagopal becoming the first ever BJP member to win a seat in the Kerala state assembly in 2016.\footnote{`Kerala Assembly Results 2016: As it Happened', \textit{ZeeNews}, 2016 \url{http://zeenews.india.com/live-updates/kerala-election-results-2016-1886323}} While this is only a single seat in a statewide legislative body, the effect of communal flare-ups on the Kerala state legislative body is limited by the state's unique religious demographics: 54\% Hindu, 27\% Muslim, and a very economically powerful 18\% Christian.\footnote{2011 Census of India} Kannur, a district in the northern part of the state, has been a hotbed of political violence between the right-wing RSS and the left wing CPI(M) (Communist Party of India, Marxist) since the 1990s with around 120 political murders since 1991. From 2000-2016, 31 RSS workers and 30 CPI(M) workers have been murdered.\footnote{`200 dead in five decades, but political violence still haunts Kerala's Kannur,' \textit{The Print}, June 2018 \url{https://theprint.in/pageturner/excerpt/decoding-the-roots-of-violence-in-kannur-kerala/69065/}} Since Kerala's development indicators are by far the highest in India, the Hindu right-wing cannot couch their appeals on economic grounds. But they cannot overcome the ground realities of demography either, so they must find another way to appeal to the electorate in Kerala. How they are managing to do this, and how they are managing to mobilize supporters into violence against the traditionally strong Kerala left would be a natural extension to the project begun in this essay. 
\end{document}